\section{Обзор предметной области}
Сперва уточним определение понятие ''зеркало''. Общепринятым является, что сайт $B$ является зеркалом сайта $A$, если существует достаточно большое множество страниц на сайте $A$, что для любой страницы из этого множества существует \textit{очень похожая} страница на сайте $B$. Степень похожести — весьма субъективная оценка, и устанавливается различными способами, например, выбором ключевых слов со страниц и их сравнение.

Можно заметить, что ''зеркало'' — это отношение эквивалентности. Действительно:
\begin{enumerate}
\item $A$ является зеркалом самого себя.
\item если $A$ является зеркалом $B$, то и $B$ — зеркало $A$.
\item если $A$ зеркало $B$, и $B$ зеркало $C$, то $A$ зеркало $C$.
\end{enumerate}

Большинство алгоритмов поиска зеркал действуют следующим образом:
\begin{enumerate}
\item получают на вход список страниц (URL-адресов) с различных сайтов;
\item оставляют в этом списке только сайты, в которых доступно достаточное количество страниц, например, более ста;
\item запрашивают различную информацию:
\begin{itemize}
\item IP-адрес сервера
\item миниатюру страницы
\item список ссылок с этой страницы на другие
\item ...
\end{itemize}
\item составляют пары сайтов-кандидатов на зеркала;
\item анализ пар на основе имеющейся информации.
\end{enumerate}

Для анализа пар применяются следующие методы:
\begin{itemize}
\item Сравнение IP-адресов. Например, для $IPv4$, если IP-адреса равны, либо равны первые три октета, то сайты считаются зеркалами.
\item Сравнение URL-адресов. Например, если к двум сайтам с различными префиксами адресов дописывать суффиксы, и при этом страницы на полученных адресах будут похожи, то сайты являются зеркалами.
\item Сравнение списка ссылок со страниц. Например, если с двух сайтов ссылки ведут на одно и то же множество страниц, то эти сайты являются зеркалами.
\item Сравнение карты сайтов. Например, если два дерева сайтов практически совпадают, то сайты являются зеркалами.
\end{itemize}