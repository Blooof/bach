\section{Обзор предметной области}
\subsection{Определение}
Сперва уточним определение понятие ''зеркало''. Общепринятым является, что сайт
$B$ является зеркалом сайта $A$, если существует достаточно большое множество
страниц на сайте $A$, что для любой страницы из этого множества существует
\textit{очень похожая} страница на сайте $B$. Степень похожести — весьма
субъективная оценка, и устанавливается различными способами, например, выбором
ключевых слов со страниц и их сравнение.

Можно заметить, что ''зеркало'' — это отношение эквивалентности. Действительно:
\begin{enumerate}
\item $A$ является зеркалом самого себя.
\item если $A$ является зеркалом $B$, то и $B$ — зеркало $A$.
\item если $A$ зеркало $B$, и $B$ зеркало $C$, то $A$ зеркало $C$.
\end{enumerate}

\subsection{Проблемы}
При поиске зеркал мы сталкиваемся с несколькими проблемами:
\begin{itemize}
\item чаще всего доступен только список с URL-адресами страниц, полученный "веб-пауком", сервером или каким-то другим образом;
\item часть страниц может измениться, устареть или быть недоступна;
\item получить/хранить страницы (сайты) целиком может оказаться затруднительно (например, не хватает оперативной памяти).
\item ...
\end{itemize}

\subsection{Известные решения}
Большинство алгоритмов поиска зеркал действуют следующим образом:
\begin{enumerate}
\item получают на вход список страниц (URL-адресов) с различных сайтов;
\item оставляют в этом списке только сайты, в которых доступно достаточное
количество страниц, например, более ста;
\item запрашивают различную информацию:
\begin{itemize}
\item IP-адрес сервера;
\item миниатюру страницы;
\item список ссылок с этой страницы на другие;
\item ...
\end{itemize}
\item составляют пары сайтов-кандидатов на зеркала;
\item производят анализ пар на основе имеющейся информации.
\end{enumerate}

Для анализа пар применяются следующие методы:
\begin{itemize}
\item Сравнение IP-адресов. Например, для $IPv4$, можно сравнивать целиком
IP-адреса, либо какие-то части, например, первые три октета.
\item Сравнение URL-адресов. Например, можно дописывать к префиксам ссылок
различные суффиксы и сравнивать страницы, на которые ведут получившиеся ссылки.
\item Сравнение списка ссылок с сайтов на другие. То есть можно оценивать, ведут
ли ссылки с проверяемых сайтов друг на друга или на одни и те же сайты.
\item Сравнение карты сайтов.
\item Составление правил преобразования ссылок одного сайта в ссылки другого. 
\item ...
\end{itemize}

Далее, на основе результатов этого анализа, алгоритмы делают различные выводы.
Например, если для каждого суффикса $P$ на сайте $A$ (http://$A$/$P$) есть похожий
документ на сайте $B$ (http://$B$/$P$) то сайты с некоторой вероятностью являются зеркалами. Проверка всех
условий может занять длительное время, особенно если список сайтов велик.
Поэтому стоит проверять не все, а выделить или придумать наиболее значимые. Наша
цель состоит в том, чтобы предложить достаточные условия, после проверки которых
можно было бы с высокой точностью ответить, являются ли сайты клонами. 
